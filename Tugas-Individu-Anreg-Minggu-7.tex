% Options for packages loaded elsewhere
\PassOptionsToPackage{unicode}{hyperref}
\PassOptionsToPackage{hyphens}{url}
%
\documentclass[
]{article}
\usepackage{amsmath,amssymb}
\usepackage{iftex}
\ifPDFTeX
  \usepackage[T1]{fontenc}
  \usepackage[utf8]{inputenc}
  \usepackage{textcomp} % provide euro and other symbols
\else % if luatex or xetex
  \usepackage{unicode-math} % this also loads fontspec
  \defaultfontfeatures{Scale=MatchLowercase}
  \defaultfontfeatures[\rmfamily]{Ligatures=TeX,Scale=1}
\fi
\usepackage{lmodern}
\ifPDFTeX\else
  % xetex/luatex font selection
\fi
% Use upquote if available, for straight quotes in verbatim environments
\IfFileExists{upquote.sty}{\usepackage{upquote}}{}
\IfFileExists{microtype.sty}{% use microtype if available
  \usepackage[]{microtype}
  \UseMicrotypeSet[protrusion]{basicmath} % disable protrusion for tt fonts
}{}
\makeatletter
\@ifundefined{KOMAClassName}{% if non-KOMA class
  \IfFileExists{parskip.sty}{%
    \usepackage{parskip}
  }{% else
    \setlength{\parindent}{0pt}
    \setlength{\parskip}{6pt plus 2pt minus 1pt}}
}{% if KOMA class
  \KOMAoptions{parskip=half}}
\makeatother
\usepackage{xcolor}
\usepackage[margin=1in]{geometry}
\usepackage{color}
\usepackage{fancyvrb}
\newcommand{\VerbBar}{|}
\newcommand{\VERB}{\Verb[commandchars=\\\{\}]}
\DefineVerbatimEnvironment{Highlighting}{Verbatim}{commandchars=\\\{\}}
% Add ',fontsize=\small' for more characters per line
\usepackage{framed}
\definecolor{shadecolor}{RGB}{248,248,248}
\newenvironment{Shaded}{\begin{snugshade}}{\end{snugshade}}
\newcommand{\AlertTok}[1]{\textcolor[rgb]{0.94,0.16,0.16}{#1}}
\newcommand{\AnnotationTok}[1]{\textcolor[rgb]{0.56,0.35,0.01}{\textbf{\textit{#1}}}}
\newcommand{\AttributeTok}[1]{\textcolor[rgb]{0.13,0.29,0.53}{#1}}
\newcommand{\BaseNTok}[1]{\textcolor[rgb]{0.00,0.00,0.81}{#1}}
\newcommand{\BuiltInTok}[1]{#1}
\newcommand{\CharTok}[1]{\textcolor[rgb]{0.31,0.60,0.02}{#1}}
\newcommand{\CommentTok}[1]{\textcolor[rgb]{0.56,0.35,0.01}{\textit{#1}}}
\newcommand{\CommentVarTok}[1]{\textcolor[rgb]{0.56,0.35,0.01}{\textbf{\textit{#1}}}}
\newcommand{\ConstantTok}[1]{\textcolor[rgb]{0.56,0.35,0.01}{#1}}
\newcommand{\ControlFlowTok}[1]{\textcolor[rgb]{0.13,0.29,0.53}{\textbf{#1}}}
\newcommand{\DataTypeTok}[1]{\textcolor[rgb]{0.13,0.29,0.53}{#1}}
\newcommand{\DecValTok}[1]{\textcolor[rgb]{0.00,0.00,0.81}{#1}}
\newcommand{\DocumentationTok}[1]{\textcolor[rgb]{0.56,0.35,0.01}{\textbf{\textit{#1}}}}
\newcommand{\ErrorTok}[1]{\textcolor[rgb]{0.64,0.00,0.00}{\textbf{#1}}}
\newcommand{\ExtensionTok}[1]{#1}
\newcommand{\FloatTok}[1]{\textcolor[rgb]{0.00,0.00,0.81}{#1}}
\newcommand{\FunctionTok}[1]{\textcolor[rgb]{0.13,0.29,0.53}{\textbf{#1}}}
\newcommand{\ImportTok}[1]{#1}
\newcommand{\InformationTok}[1]{\textcolor[rgb]{0.56,0.35,0.01}{\textbf{\textit{#1}}}}
\newcommand{\KeywordTok}[1]{\textcolor[rgb]{0.13,0.29,0.53}{\textbf{#1}}}
\newcommand{\NormalTok}[1]{#1}
\newcommand{\OperatorTok}[1]{\textcolor[rgb]{0.81,0.36,0.00}{\textbf{#1}}}
\newcommand{\OtherTok}[1]{\textcolor[rgb]{0.56,0.35,0.01}{#1}}
\newcommand{\PreprocessorTok}[1]{\textcolor[rgb]{0.56,0.35,0.01}{\textit{#1}}}
\newcommand{\RegionMarkerTok}[1]{#1}
\newcommand{\SpecialCharTok}[1]{\textcolor[rgb]{0.81,0.36,0.00}{\textbf{#1}}}
\newcommand{\SpecialStringTok}[1]{\textcolor[rgb]{0.31,0.60,0.02}{#1}}
\newcommand{\StringTok}[1]{\textcolor[rgb]{0.31,0.60,0.02}{#1}}
\newcommand{\VariableTok}[1]{\textcolor[rgb]{0.00,0.00,0.00}{#1}}
\newcommand{\VerbatimStringTok}[1]{\textcolor[rgb]{0.31,0.60,0.02}{#1}}
\newcommand{\WarningTok}[1]{\textcolor[rgb]{0.56,0.35,0.01}{\textbf{\textit{#1}}}}
\usepackage{graphicx}
\makeatletter
\def\maxwidth{\ifdim\Gin@nat@width>\linewidth\linewidth\else\Gin@nat@width\fi}
\def\maxheight{\ifdim\Gin@nat@height>\textheight\textheight\else\Gin@nat@height\fi}
\makeatother
% Scale images if necessary, so that they will not overflow the page
% margins by default, and it is still possible to overwrite the defaults
% using explicit options in \includegraphics[width, height, ...]{}
\setkeys{Gin}{width=\maxwidth,height=\maxheight,keepaspectratio}
% Set default figure placement to htbp
\makeatletter
\def\fps@figure{htbp}
\makeatother
\setlength{\emergencystretch}{3em} % prevent overfull lines
\providecommand{\tightlist}{%
  \setlength{\itemsep}{0pt}\setlength{\parskip}{0pt}}
\setcounter{secnumdepth}{-\maxdimen} % remove section numbering
\ifLuaTeX
  \usepackage{selnolig}  % disable illegal ligatures
\fi
\IfFileExists{bookmark.sty}{\usepackage{bookmark}}{\usepackage{hyperref}}
\IfFileExists{xurl.sty}{\usepackage{xurl}}{} % add URL line breaks if available
\urlstyle{same}
\hypersetup{
  pdftitle={Tugas Individu Anreg Minggu 7},
  pdfauthor={Dalilah Husna},
  hidelinks,
  pdfcreator={LaTeX via pandoc}}

\title{Tugas Individu Anreg Minggu 7}
\author{Dalilah Husna}
\date{2024-03-05}

\begin{document}
\maketitle

\hypertarget{membaca-data}{%
\section{Membaca data}\label{membaca-data}}

\begin{Shaded}
\begin{Highlighting}[]
\FunctionTok{library}\NormalTok{(readxl)}
\NormalTok{data }\OtherTok{\textless{}{-}} \FunctionTok{read\_xlsx}\NormalTok{(}\StringTok{"C:/Users/USER/OneDrive {-} apps.ipb.ac.id/Semester 4/Analisis Regresi/UTS/Pertemuan 7/Tugas Individu Pertemuan 7.xlsx"}\NormalTok{)}
\NormalTok{data}
\end{Highlighting}
\end{Shaded}

\begin{verbatim}
## # A tibble: 15 x 2
##        X     Y
##    <dbl> <dbl>
##  1     2    54
##  2     5    50
##  3     7    45
##  4    10    37
##  5    14    35
##  6    19    25
##  7    26    20
##  8    31    16
##  9    34    18
## 10    38    13
## 11    45     8
## 12    52    11
## 13    53     8
## 14    60     4
## 15    65     6
\end{verbatim}

\begin{Shaded}
\begin{Highlighting}[]
\FunctionTok{plot}\NormalTok{(data}\SpecialCharTok{$}\NormalTok{X,data}\SpecialCharTok{$}\NormalTok{Y)}
\end{Highlighting}
\end{Shaded}

\includegraphics{Tugas-Individu-Anreg-Minggu-7_files/figure-latex/unnamed-chunk-2-1.pdf}

\hypertarget{model-linier}{%
\section{Model linier}\label{model-linier}}

\begin{Shaded}
\begin{Highlighting}[]
\NormalTok{modelreg }\OtherTok{=} \FunctionTok{lm}\NormalTok{(}\AttributeTok{formula=}\NormalTok{Y}\SpecialCharTok{\textasciitilde{}}\NormalTok{., }\AttributeTok{data=}\NormalTok{data)}
\FunctionTok{summary}\NormalTok{(modelreg)}
\end{Highlighting}
\end{Shaded}

\begin{verbatim}
## 
## Call:
## lm(formula = Y ~ ., data = data)
## 
## Residuals:
##     Min      1Q  Median      3Q     Max 
## -7.1628 -4.7313 -0.9253  3.7386  9.0446 
## 
## Coefficients:
##             Estimate Std. Error t value Pr(>|t|)    
## (Intercept) 46.46041    2.76218   16.82 3.33e-10 ***
## X           -0.75251    0.07502  -10.03 1.74e-07 ***
## ---
## Signif. codes:  0 '***' 0.001 '**' 0.01 '*' 0.05 '.' 0.1 ' ' 1
## 
## Residual standard error: 5.891 on 13 degrees of freedom
## Multiple R-squared:  0.8856, Adjusted R-squared:  0.8768 
## F-statistic: 100.6 on 1 and 13 DF,  p-value: 1.736e-07
\end{verbatim}

\begin{Shaded}
\begin{Highlighting}[]
\NormalTok{modelreg}
\end{Highlighting}
\end{Shaded}

\begin{verbatim}
## 
## Call:
## lm(formula = Y ~ ., data = data)
## 
## Coefficients:
## (Intercept)            X  
##     46.4604      -0.7525
\end{verbatim}

\hypertarget{eksplorasi-kondisi-gauss-markov-pemeriksaan-dengan-grafik}{%
\section{Eksplorasi kondisi Gauss-Markov, pemeriksaan dengan
grafik}\label{eksplorasi-kondisi-gauss-markov-pemeriksaan-dengan-grafik}}

\hypertarget{plot-sisaan-vs-y-duga}{%
\subsection{1. Plot sisaan vs Y duga}\label{plot-sisaan-vs-y-duga}}

\begin{Shaded}
\begin{Highlighting}[]
\FunctionTok{plot}\NormalTok{(modelreg,}\DecValTok{1}\NormalTok{)}
\end{Highlighting}
\end{Shaded}

\includegraphics{Tugas-Individu-Anreg-Minggu-7_files/figure-latex/unnamed-chunk-4-1.pdf}

\hypertarget{plot-sisaan-vs-urutan}{%
\subsection{2. Plot sisaan vs urutan}\label{plot-sisaan-vs-urutan}}

\begin{Shaded}
\begin{Highlighting}[]
\FunctionTok{plot}\NormalTok{(}\AttributeTok{x=}\DecValTok{1}\SpecialCharTok{:}\FunctionTok{dim}\NormalTok{(data)[}\DecValTok{1}\NormalTok{],}
     \AttributeTok{y=}\NormalTok{modelreg}\SpecialCharTok{$}\NormalTok{residuals,}
     \AttributeTok{type=}\StringTok{\textquotesingle{}b\textquotesingle{}}\NormalTok{,}
     \AttributeTok{ylab=}\StringTok{"residuals"}\NormalTok{,}
     \AttributeTok{xlab=}\StringTok{"observation"}\NormalTok{)}
\end{Highlighting}
\end{Shaded}

\includegraphics{Tugas-Individu-Anreg-Minggu-7_files/figure-latex/unnamed-chunk-5-1.pdf}

\hypertarget{normalitas-sisaan}{%
\subsection{3. Normalitas sisaan}\label{normalitas-sisaan}}

\begin{Shaded}
\begin{Highlighting}[]
\FunctionTok{plot}\NormalTok{(modelreg,}\DecValTok{2}\NormalTok{)}
\end{Highlighting}
\end{Shaded}

\includegraphics{Tugas-Individu-Anreg-Minggu-7_files/figure-latex/unnamed-chunk-6-1.pdf}

\hypertarget{pengujian-asumsi}{%
\section{Pengujian Asumsi}\label{pengujian-asumsi}}

\hypertarget{a.-kondisi-gauss-markov}{%
\subsection{A. Kondisi Gauss-Markov}\label{a.-kondisi-gauss-markov}}

\hypertarget{nilai-harapan-sisaan-sama-dengan-0}{%
\subsubsection{1. Nilai harapan sisaan sama dengan
0}\label{nilai-harapan-sisaan-sama-dengan-0}}

\begin{Shaded}
\begin{Highlighting}[]
\FunctionTok{t.test}\NormalTok{(modelreg}\SpecialCharTok{$}\NormalTok{residuals,}\AttributeTok{mu=}\DecValTok{0}\NormalTok{,}\AttributeTok{conf.level=}\FloatTok{0.95}\NormalTok{)}
\end{Highlighting}
\end{Shaded}

\begin{verbatim}
## 
##  One Sample t-test
## 
## data:  modelreg$residuals
## t = -4.9493e-16, df = 14, p-value = 1
## alternative hypothesis: true mean is not equal to 0
## 95 percent confidence interval:
##  -3.143811  3.143811
## sample estimates:
##     mean of x 
## -7.254614e-16
\end{verbatim}

Hipotesis yang diuji : H0 : miu = 0 H1 : miu tidak sama dengan 0

Dari uji t diatas didapat nilai-p sebesar 1. Dengan alpha sebesar 0,05
maka nilai-p lebih besar dari alpha. Dapat disimpulkan bahwa tak tolak
H0. Nilai harapan sisaan sama dengan 0.

\hypertarget{ragam-sisaan-homogen}{%
\subsubsection{2. Ragam sisaan homogen}\label{ragam-sisaan-homogen}}

\begin{Shaded}
\begin{Highlighting}[]
\FunctionTok{library}\NormalTok{(lmtest)}
\end{Highlighting}
\end{Shaded}

\begin{verbatim}
## Loading required package: zoo
\end{verbatim}

\begin{verbatim}
## 
## Attaching package: 'zoo'
\end{verbatim}

\begin{verbatim}
## The following objects are masked from 'package:base':
## 
##     as.Date, as.Date.numeric
\end{verbatim}

\begin{Shaded}
\begin{Highlighting}[]
\NormalTok{homogen }\OtherTok{\textless{}{-}} \FunctionTok{bptest}\NormalTok{(modelreg)}
\NormalTok{homogen}
\end{Highlighting}
\end{Shaded}

\begin{verbatim}
## 
##  studentized Breusch-Pagan test
## 
## data:  modelreg
## BP = 0.52819, df = 1, p-value = 0.4674
\end{verbatim}

Hipotesis yang diuji : H0 : Ragam sisaan homogen H1 : Ragam sisaan
heterogen

Dari uji Breusch-Pagan guna melihat apakah ragam sisaan homogen didapat
bahwa nilai-p sebesar 0,4674. Karena nilai-p \textgreater{} alpha 0,05
maka tak tolak H0. Maka ragam sisaan homogen.

\hypertarget{sisaan-saling-bebas}{%
\subsubsection{3. Sisaan saling bebas}\label{sisaan-saling-bebas}}

\begin{Shaded}
\begin{Highlighting}[]
\FunctionTok{library}\NormalTok{(randtests)}
\FunctionTok{runs.test}\NormalTok{(modelreg}\SpecialCharTok{$}\NormalTok{residuals)}
\end{Highlighting}
\end{Shaded}

\begin{verbatim}
## 
##  Runs Test
## 
## data:  modelreg$residuals
## statistic = -2.7817, runs = 3, n1 = 7, n2 = 7, n = 14, p-value =
## 0.005407
## alternative hypothesis: nonrandomness
\end{verbatim}

Hipotesis yang diuji : H0 : Sisaan saling bebas H1 : Sisaan tidak saling
bebas

Dari uji diatas, didapat nilai-p sebesar 0,005407. Dengan alpha sebesar
0,05 maka dapat disimpulkan tolak H0 karena nilai-p \textless{} alpha.
Maka sisaan tidak saling bebas.

\hypertarget{b.-galat-menyebar-normal}{%
\subsection{B. Galat menyebar normal}\label{b.-galat-menyebar-normal}}

\begin{Shaded}
\begin{Highlighting}[]
\FunctionTok{library}\NormalTok{(nortest)}
\NormalTok{sisaan\_model }\OtherTok{\textless{}{-}} \FunctionTok{resid}\NormalTok{(modelreg)}
\NormalTok{(norm\_model }\OtherTok{\textless{}{-}} \FunctionTok{lillie.test}\NormalTok{(sisaan\_model))}
\end{Highlighting}
\end{Shaded}

\begin{verbatim}
## 
##  Lilliefors (Kolmogorov-Smirnov) normality test
## 
## data:  sisaan_model
## D = 0.12432, p-value = 0.7701
\end{verbatim}

Hipotesis yang diuji : H0 : galat menyebar normal H1 : galat menyebar
tidak normal

Dari uji Kolmogrov-Smirnov diatas, didapat nilai-p sebesar 0,7701 Dengan
alpha sebesar 0,05, maka dapat kita simpulkan untuk tak tolak H0 karena
nilai-p \textgreater{} 0,05. Maka sisaan menyebar normal.

\hypertarget{c.-galat-bebas-terhadap-peubah-bebas}{%
\subsection{C. Galat bebas terhadap peubah
bebas}\label{c.-galat-bebas-terhadap-peubah-bebas}}

\begin{Shaded}
\begin{Highlighting}[]
\NormalTok{model }\OtherTok{\textless{}{-}} \FunctionTok{lm}\NormalTok{(Y}\SpecialCharTok{\textasciitilde{}}\NormalTok{., }\AttributeTok{data =}\NormalTok{ data)}
\FunctionTok{summary}\NormalTok{(model)}
\end{Highlighting}
\end{Shaded}

\begin{verbatim}
## 
## Call:
## lm(formula = Y ~ ., data = data)
## 
## Residuals:
##     Min      1Q  Median      3Q     Max 
## -7.1628 -4.7313 -0.9253  3.7386  9.0446 
## 
## Coefficients:
##             Estimate Std. Error t value Pr(>|t|)    
## (Intercept) 46.46041    2.76218   16.82 3.33e-10 ***
## X           -0.75251    0.07502  -10.03 1.74e-07 ***
## ---
## Signif. codes:  0 '***' 0.001 '**' 0.01 '*' 0.05 '.' 0.1 ' ' 1
## 
## Residual standard error: 5.891 on 13 degrees of freedom
## Multiple R-squared:  0.8856, Adjusted R-squared:  0.8768 
## F-statistic: 100.6 on 1 and 13 DF,  p-value: 1.736e-07
\end{verbatim}

Hipotesis yang diuji : H0 : galat bebas terhadap peubah bebas H1 : galat
tidak bebas terhadap peubah bebas

Dari ANOVA diatas dapat dilihat bahwa nilai-p sebesar 1,736e-07. Dengan
alpha sebesar 0,05, dapat dikatakan bahwa tak tolak H0. Maka galat bebas
terhadap peubah bebas.

Maka dari sederetan uji diatas asumsi yang tidak terpenuhi adalah sisaan
saling bebas. Karena ada satu asumsi yang tidak terpenuhi maka
diperlukan penanganan terhadap kondisi yang tidak standar.

\hypertarget{penanganan-kondisi-tak-standar}{%
\section{Penanganan kondisi tak
standar}\label{penanganan-kondisi-tak-standar}}

\begin{Shaded}
\begin{Highlighting}[]
\NormalTok{data\_x }\OtherTok{\textless{}{-}} \FunctionTok{sqrt}\NormalTok{(data}\SpecialCharTok{$}\NormalTok{X)}
\NormalTok{data\_y }\OtherTok{\textless{}{-}} \FunctionTok{sqrt}\NormalTok{(data}\SpecialCharTok{$}\NormalTok{Y)}
\NormalTok{transformasi }\OtherTok{\textless{}{-}} \FunctionTok{data.frame}\NormalTok{(data\_x,data\_y)}
\NormalTok{transformasi}
\end{Highlighting}
\end{Shaded}

\begin{verbatim}
##      data_x   data_y
## 1  1.414214 7.348469
## 2  2.236068 7.071068
## 3  2.645751 6.708204
## 4  3.162278 6.082763
## 5  3.741657 5.916080
## 6  4.358899 5.000000
## 7  5.099020 4.472136
## 8  5.567764 4.000000
## 9  5.830952 4.242641
## 10 6.164414 3.605551
## 11 6.708204 2.828427
## 12 7.211103 3.316625
## 13 7.280110 2.828427
## 14 7.745967 2.000000
## 15 8.062258 2.449490
\end{verbatim}

\hypertarget{model-linier-baru}{%
\subsection{Model Linier Baru}\label{model-linier-baru}}

\begin{Shaded}
\begin{Highlighting}[]
\NormalTok{modelbaru }\OtherTok{\textless{}{-}} \FunctionTok{lm}\NormalTok{(transformasi}\SpecialCharTok{$}\NormalTok{data\_y}\SpecialCharTok{\textasciitilde{}}\NormalTok{transformasi}\SpecialCharTok{$}\NormalTok{data\_x, }\AttributeTok{data=}\NormalTok{transformasi)}
\FunctionTok{summary}\NormalTok{(modelbaru)}
\end{Highlighting}
\end{Shaded}

\begin{verbatim}
## 
## Call:
## lm(formula = transformasi$data_y ~ transformasi$data_x, data = transformasi)
## 
## Residuals:
##      Min       1Q   Median       3Q      Max 
## -0.42765 -0.17534 -0.05753  0.21223  0.46960 
## 
## Coefficients:
##                     Estimate Std. Error t value Pr(>|t|)    
## (Intercept)          8.71245    0.19101   45.61 9.83e-16 ***
## transformasi$data_x -0.81339    0.03445  -23.61 4.64e-12 ***
## ---
## Signif. codes:  0 '***' 0.001 '**' 0.01 '*' 0.05 '.' 0.1 ' ' 1
## 
## Residual standard error: 0.2743 on 13 degrees of freedom
## Multiple R-squared:  0.9772, Adjusted R-squared:  0.9755 
## F-statistic: 557.3 on 1 and 13 DF,  p-value: 4.643e-12
\end{verbatim}

\begin{Shaded}
\begin{Highlighting}[]
\NormalTok{modelbaru}
\end{Highlighting}
\end{Shaded}

\begin{verbatim}
## 
## Call:
## lm(formula = transformasi$data_y ~ transformasi$data_x, data = transformasi)
## 
## Coefficients:
##         (Intercept)  transformasi$data_x  
##              8.7125              -0.8134
\end{verbatim}

\begin{Shaded}
\begin{Highlighting}[]
\FunctionTok{plot}\NormalTok{(transformasi}\SpecialCharTok{$}\NormalTok{data\_x,transformasi}\SpecialCharTok{$}\NormalTok{data\_y)}
\end{Highlighting}
\end{Shaded}

\includegraphics{Tugas-Individu-Anreg-Minggu-7_files/figure-latex/unnamed-chunk-14-1.pdf}

\hypertarget{eksplorasi-kondisi-gauss-markov-pemeriksaan-dengan-grafik-untuk-model-baru}{%
\section{Eksplorasi kondisi Gauss-Markov, pemeriksaan dengan grafik
untuk model
baru}\label{eksplorasi-kondisi-gauss-markov-pemeriksaan-dengan-grafik-untuk-model-baru}}

\hypertarget{plot-sisaan-vs-y-duga-1}{%
\subsection{1. Plot sisaan vs Y duga}\label{plot-sisaan-vs-y-duga-1}}

\begin{Shaded}
\begin{Highlighting}[]
\FunctionTok{plot}\NormalTok{(modelbaru,}\DecValTok{1}\NormalTok{)}
\end{Highlighting}
\end{Shaded}

\includegraphics{Tugas-Individu-Anreg-Minggu-7_files/figure-latex/unnamed-chunk-15-1.pdf}

\hypertarget{plot-sisaan-vs-urutan-1}{%
\subsection{2. Plot sisaan vs urutan}\label{plot-sisaan-vs-urutan-1}}

\begin{Shaded}
\begin{Highlighting}[]
\FunctionTok{plot}\NormalTok{(}\AttributeTok{x =} \DecValTok{1}\SpecialCharTok{:}\FunctionTok{dim}\NormalTok{(transformasi)[}\DecValTok{1}\NormalTok{],}
     \AttributeTok{y =}\NormalTok{ modelbaru}\SpecialCharTok{$}\NormalTok{residuals,}
     \AttributeTok{type =} \StringTok{\textquotesingle{}b\textquotesingle{}}\NormalTok{, }
     \AttributeTok{ylab =} \StringTok{"Residuals"}\NormalTok{,}
     \AttributeTok{xlab =} \StringTok{"Observation"}\NormalTok{)}
\end{Highlighting}
\end{Shaded}

\includegraphics{Tugas-Individu-Anreg-Minggu-7_files/figure-latex/unnamed-chunk-16-1.pdf}

\hypertarget{eksplorasi-normalitas-sisaan}{%
\subsection{3. Eksplorasi normalitas
sisaan}\label{eksplorasi-normalitas-sisaan}}

\begin{Shaded}
\begin{Highlighting}[]
\FunctionTok{plot}\NormalTok{(modelbaru,}\DecValTok{2}\NormalTok{)}
\end{Highlighting}
\end{Shaded}

\includegraphics{Tugas-Individu-Anreg-Minggu-7_files/figure-latex/unnamed-chunk-17-1.pdf}

\hypertarget{uji-formal-kondisi-gauss-markov}{%
\section{Uji formal kondisi
Gauss-Markov}\label{uji-formal-kondisi-gauss-markov}}

\hypertarget{nilai-harapan-sisaan-sama-dengan-0-1}{%
\subsection{1. Nilai harapan sisaan sama dengan
0}\label{nilai-harapan-sisaan-sama-dengan-0-1}}

\begin{Shaded}
\begin{Highlighting}[]
\NormalTok{nilaiharapan}\OtherTok{\textless{}{-}} \FunctionTok{t.test}\NormalTok{(modelbaru}\SpecialCharTok{$}\NormalTok{residuals,}\AttributeTok{mu =} \DecValTok{0}\NormalTok{,}\AttributeTok{conf.level =} \FloatTok{0.95}\NormalTok{)}
\NormalTok{nilaiharapan}
\end{Highlighting}
\end{Shaded}

\begin{verbatim}
## 
##  One Sample t-test
## 
## data:  modelbaru$residuals
## t = 2.0334e-16, df = 14, p-value = 1
## alternative hypothesis: true mean is not equal to 0
## 95 percent confidence interval:
##  -0.1463783  0.1463783
## sample estimates:
##    mean of x 
## 1.387779e-17
\end{verbatim}

\begin{Shaded}
\begin{Highlighting}[]
\FunctionTok{ifelse}\NormalTok{(nilaiharapan}\SpecialCharTok{$}\NormalTok{p.value }\SpecialCharTok{\textless{}} \FloatTok{0.05}\NormalTok{, }\StringTok{"Nilai harapan tidak sama dengan 0"}\NormalTok{, }\StringTok{"Nilai harapan sama dengan 0"}\NormalTok{)}
\end{Highlighting}
\end{Shaded}

\begin{verbatim}
## [1] "Nilai harapan sama dengan 0"
\end{verbatim}

\hypertarget{ragam-sisaan-homogen-1}{%
\subsection{2. Ragam sisaan homogen}\label{ragam-sisaan-homogen-1}}

\begin{Shaded}
\begin{Highlighting}[]
\FunctionTok{library}\NormalTok{(car)}
\end{Highlighting}
\end{Shaded}

\begin{verbatim}
## Loading required package: carData
\end{verbatim}

\begin{Shaded}
\begin{Highlighting}[]
\NormalTok{ujihomogen }\OtherTok{\textless{}{-}} \FunctionTok{ncvTest}\NormalTok{(modelbaru)}
\NormalTok{ujihomogen}
\end{Highlighting}
\end{Shaded}

\begin{verbatim}
## Non-constant Variance Score Test 
## Variance formula: ~ fitted.values 
## Chisquare = 2.160411, Df = 1, p = 0.14161
\end{verbatim}

\begin{Shaded}
\begin{Highlighting}[]
\FunctionTok{ifelse}\NormalTok{(ujihomogen}\SpecialCharTok{$}\NormalTok{p }\SpecialCharTok{\textless{}} \FloatTok{0.05}\NormalTok{, }\StringTok{"Ragam Tidak Homogen"}\NormalTok{, }\StringTok{"Ragam Homogen"}\NormalTok{)}
\end{Highlighting}
\end{Shaded}

\begin{verbatim}
## [1] "Ragam Homogen"
\end{verbatim}

\hypertarget{sisaan-saling-bebas-1}{%
\subsection{3. Sisaan saling bebas}\label{sisaan-saling-bebas-1}}

\begin{Shaded}
\begin{Highlighting}[]
\NormalTok{salingbebas}\OtherTok{\textless{}{-}}\FunctionTok{runs.test}\NormalTok{(modelbaru}\SpecialCharTok{$}\NormalTok{residuals)}
\NormalTok{salingbebas}
\end{Highlighting}
\end{Shaded}

\begin{verbatim}
## 
##  Runs Test
## 
## data:  modelbaru$residuals
## statistic = 0, runs = 8, n1 = 7, n2 = 7, n = 14, p-value = 1
## alternative hypothesis: nonrandomness
\end{verbatim}

\begin{Shaded}
\begin{Highlighting}[]
\FunctionTok{ifelse}\NormalTok{(salingbebas}\SpecialCharTok{$}\NormalTok{p.value }\SpecialCharTok{\textless{}} \FloatTok{0.05}\NormalTok{, }\StringTok{"Sisaan tidak saling bebas"}\NormalTok{, }\StringTok{"Sisaan saling bebas"}\NormalTok{)}
\end{Highlighting}
\end{Shaded}

\begin{verbatim}
## [1] "Sisaan saling bebas"
\end{verbatim}

\hypertarget{normalitas}{%
\subsection{Normalitas}\label{normalitas}}

\begin{Shaded}
\begin{Highlighting}[]
\NormalTok{sisaanmodel }\OtherTok{\textless{}{-}} \FunctionTok{resid}\NormalTok{(modelbaru)}
\NormalTok{(normmodel }\OtherTok{\textless{}{-}} \FunctionTok{lillie.test}\NormalTok{(sisaanmodel))}
\end{Highlighting}
\end{Shaded}

\begin{verbatim}
## 
##  Lilliefors (Kolmogorov-Smirnov) normality test
## 
## data:  sisaanmodel
## D = 0.11948, p-value = 0.817
\end{verbatim}

\begin{Shaded}
\begin{Highlighting}[]
\FunctionTok{ifelse}\NormalTok{(normmodel}\SpecialCharTok{$}\NormalTok{p.value }\SpecialCharTok{\textless{}} \FloatTok{0.05}\NormalTok{, }\StringTok{"Galat tidak menyebar normal"}\NormalTok{, }\StringTok{"Galat menyebar normal"}\NormalTok{)}
\end{Highlighting}
\end{Shaded}

\begin{verbatim}
## [1] "Galat menyebar normal"
\end{verbatim}

Maka model terbaik didapat dari hasil transformasi kedua variabel yaitu
X dan Y. Kedua variabel tersebut diakarkan, kemudian ketika diuji lagi
untuk pengujian asumsi seluruh asumsi terpenuhi. Maka, telah didapat
model terbaik serta memenuhi seluruh asumsi yang dibutuhkan. Model
regresinya menjadi : \[ \hat Y = 8.7125-0.8134X \]

\end{document}
